                                       
% This template was created for future students affiliated with the Department of Economics at the University of Bergamo by:
% - Manuel Donà (Energy Market Modeler at Falck Renewables SpA);
% - Andrea Lopes (PhD Candidate in Economics at University of Milan);
% - Andrea Zoccatelli (Data Scientist at GroupM).




\documentclass{beamer}
%\hypersetup{pdfpagemode=FullScreen} %full screen mode
\setbeamertemplate{navigation symbols}{}


\usepackage[english]{babel} 
\usepackage[latin1]{inputenc} 
\usepackage{amsmath,amssymb}
\usepackage[T1]{fontenc} 
\usepackage{curves}
\usepackage{verbatim}
\usepackage{multimedia}
\usepackage{mathptmx}
\usepackage{graphicx}
\usepackage{booktabs}
\usepackage{hyperref}
\usepackage{multirow}
\usepackage{xcolor}
\usepackage{algorithm,algorithmic}
\renewcommand{\algorithmicrequire}{\textbf{Input:}}
\renewcommand{\algorithmicensure}{\textbf{Output:}}
\usepackage{lipsum}
\setbeamertemplate{caption}[numbered] % For numbering figures

\mode<presentation> {

\usetheme{Madrid} 

\usecolortheme{whale} 

%\setbeamertemplate{footline} % To remove the footer line in all slides uncomment this line


\setbeamertemplate{navigation symbols}{} % To remove the navigation symbols from the bottom of all slides uncomment this line
}

\usepackage{graphicx}
\usepackage{booktabs} 
\setbeamercovered{transparent}
\setbeamertemplate{bibliography item}[text]
\setbeamertemplate{theorems}[numbered]
\setbeamerfont{title}{size=\Large}%\miniscule,command,tiny, scriptsize,footnotesize,small,normalsize,large,Large,LARGE,huge,Huge,HUGE
\setbeamerfont{date}{size=\tiny}%{\fontsize{40}{48} \selectfont Text}

\setbeamertemplate{itemize items}[ball] % if you want a ball
\setbeamertemplate{itemize subitem}[circle] % if you want a circle
\setbeamertemplate{itemize subsubitem}[triangle] % if you want a triangle

%-------customized frame-------
\newcounter{cont}

\makeatletter
%allowframebreaks numbering in the title
\setbeamertemplate{frametitle continuation}{%
   % \setcounter{cont}{\beamer@endpageofframe}%
    %\addtocounter{cont}{1}%
   % \addtocounter{cont}{-\beamer@startpageofframe}%
   % (\insertcontinuationcount/\arabic{cont})%
}
\makeatother

%--------------------------
%	TITLE PAGE
%--------------------------

\title[Biomedical Signal Analysis]{ML/DL based biomedical signal analysis and processing for disease identification and related hardware implementation } % The short title appears at the bottom of every slide, the full title is only on the title page

\author[Archana Kumari \& Shwashwat Das]{Archana Kumari \& Shwashwat Das}

\institute[STUDENT ID] %matricola
{

{\small Supervisor} \\
\medskip
{\small Prof. Sumitra Mukhopadhyay }\\
%\begin{center}
%\includegraphics[width=0.1\textwidth]{uoh.png}
%\end{center}
\vspace{0.5cm}
%\medskip
Institute of Radiophysics and Electronics\\
University of Calcutta

%\medskip
%\textit{john@smith.com} % Your email address
}
\date[\today]{\today}
%This will place the image at position "30 right/left and 120 up/down" relative to the top left corner of the current page.
%\titlegraphic{%
  %\begin{picture}(0,0)
   % \put(37,146){\makebox(0,0)[rt]{\includegraphics[width=2.5cm]{bg.png}}}
 % \end{picture}} % you can modify the UNIBG logo to center it.
% \date{insert date} % Date
 
\begin{document}
\definecolor{your_color}{HTML}{172642}
\setbeamercolor{structure}{fg=your_color}
\begin{frame}
\titlepage % Print the title page as the first slide
\setbeamertemplate{footline}[page number]

\end{frame}
\setbeamertemplate{footline}[page number]


%-------------------------------
%	PRESENTATION SLIDES
%-------------------------------

\section{Name\_1} 
\begin{frame}[allowframebreaks]
\frametitle{Agenda}
\begin{itemize}
\item {Introduction}
\vspace{0.15cm}
\item {Goals}
\vspace{0.15cm}
\item {Non Goals}
\vspace{0.15cm}
\item {Learnings}
\vspace{0.15cm}
\item {Future Updates}
\vspace{0.15cm}
\item {References}
\vspace{0.15cm}
%\item {Appendix}
\end{itemize}

\end{frame}


%---------------------------------------

%---------------------------------------------------------------
\section{Name\_2}
\begin{frame}[allowframebreaks]
\frametitle{Introduction}
\textbf{Biomedical Signal Analysis}
\vspace{0.3cm}
\begin{itemize}
\item {ML/DL techniques to process and interpret biomedical signals, aiding in the identification and monitoring of diseases.}
\vspace{0.15cm}
\item {This will offer early diagnosis and personalized treatment opportunities.}
\end{itemize}

\end{frame}

%----------------------------------------

\section{Name\_3}
\begin{frame}[allowframebreak]
\frametitle{Goals}\small

%\begin{block}{insert block name (e.g. proposition or key assumption)}
 % Write your proposition here
%\end{block} % insert in the block fundamental assumptions, propositions, lemmas, and so on, with the CORRECT block title (e.g., Proposition 1).
\begin{itemize}
    \item {Data Collection from some hospital}
    \item {Preprocessing of the obtained data}
    \item{Feature Extraction}
    \item{ML/DL Model Training}
    \item{Validation}
    \item{Disease Identification}
    \item{Integration with the hardware}
    %\item{Biomedical Signal Conditioning}
\end{itemize}
\end{frame}

%-------------------

\section{Name\_3}
\begin{frame}[allowframebreak]
\frametitle{Streched Goals}\small

%\begin{block}{insert block name (e.g. proposition or key assumption)}
 % Write your proposition here
%\end{block} % insert in the block fundamental assumptions, propositions, lemmas, and so on, with the CORRECT block title (e.g., Proposition 1).
\begin{itemize}
    
    \item{Biomedical Signal Conditioning}
    \item{Further optimize the algorithm in terms of memory consumption and its time complexity}
   % \item{}
\end{itemize}
\end{frame}

%-------------------------------

\section{Learings}
\begin{frame}[allowframebreaks]
\frametitle{Learnings}\scriptsize

\begin{itemize}
    \item {Learn the broad types of ML techniques (Unsupervised Learning, Reinforcement Learning, Regression, Classification Learning)}
    \item {Learn about Evaluation and Cross-Validation}
    \item {Learn the concepts of Linear Regression, Decision Tress and Over-fitting}
    \item {Learn the KNN algorithm}
    \item {Created the repo and updated it with the current progress}
\end{itemize}

\end{frame}

%----------------------
\section{Name\_3}
\begin{frame}[allowframebreak]
\frametitle{Future Updates}\small

%\begin{block}{insert block name (e.g. proposition or key assumption)}
 % Write your proposition here
%\end{block} % insert in the block fundamental assumptions, propositions, lemmas, and so on, with the CORRECT block title (e.g., Proposition 1).
\begin{itemize}
    \item {Learn the next module of NPTEL.}
    \item{ Set up the hardware}
    \item{Implement the models on hardware parallely}
\end{itemize}
\end{frame}

%---------------------------------------




\section{References}
\begin{frame}[allowframebreaks]
\frametitle{References}\scriptsize
\begin{itemize}
    \item {NPTEL Playlist1 - } \href{https://onlinecourses.nptel.ac.in/noc23_cs18/preview}{https://onlinecourses.nptel.ac.in/noc23_cs18/preview}
    \item {NPTEL Playlist2 - } \href{https://archive.nptel.ac.in/courses/106/105/106105152/}{https://archive.nptel.ac.in/courses/106/105/106105152/}
    \item {Link to Dataset used for learning - } \href{https://www.kaggle.com/datasets/kaushil268/disease-prediction-using-machine-learning?resource=download}{https://www.kaggle.com/datasets/kaushil268/disease-prediction-using-machine-learning?resource=download}
    \item{Link to Project's Public Repo - } \href{https://github.com/archana-9430/Final-year-project-Biomedical-Signal-Analysis/blob/main/Presentation/Presentation-1.tex}{https://github.com/archana-9430/Final-year-project-Biomedical-Signal-Analysis/blob/main/Presentation/Presentation-1.tex}
\end{itemize}


\end{frame}

%-------------------------------
%\begin{frame}
%\Huge{\centerline{Thank You}}
%\end{frame}

%-------------------------------
\begingroup
{
\setbeamercolor{background canvas}{bg=your_color}
\setbeamertemplate{footline}{}
\begin{frame}[c]
\vfill
\begin{center}
\Large\textcolor{white}{Thank you}
\end{center}
\vfill
\end{frame}
} 

\endgroup

% Choose your favorite thank you slide

%-------------------------------
\section{Appendix}
\begin{frame}{Appendix}
    
\end{frame}


\end{document}